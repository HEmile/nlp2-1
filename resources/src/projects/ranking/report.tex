\section{Report}

Again, we expect you to write about your research and empirical findings in the form of a research paper.
A few things we expect to find in your report:

\begin{itemize}
    \item explain and motivate your choice of features
    \begin{itemize}
        \item if you induced features, give an overview of the method used (again, always motivate your choices)
    \end{itemize}
    \item explain the training algorithms you implemented, use the following (where applicable) as a guidelines
    \begin{itemize}
        \item choice of learning objective
        \item choice of regularisation method
        \item choice of classifier
        \item choice of sampling procedure (e.g. when creating negative examples for pairwise classification)
        \item choice of hyperparameters (e.g. number of negative samples, or other hyperparameters associated with your choice of machine learning algorithm)
    \end{itemize}
    \item be critical about findings
\end{itemize}

You will be assessed mainly in terms of your report, but do submit a link to your code base and output translation files.
In grading your project, we will take into account the following guidelines/criteria:

\begin{itemize}
    \item theoretical description (4 points): did you convey a concrete and correct understanding of the methodology?
    \item empirical evaluation (4 points): did you succeed in the proposed empirical investigation? Is evaluation done in a proper way
    \item writing style (2 points)
\end{itemize}

%In deciding on your grade for each criterion, we will also have in mind a weighted average of different parts of your project: rich features (40\%), tuning algorithm (40\%), evaluation (20\%).
%Reviewers will take into account the 3 main tasks, namely, \emph{rich features}, \emph{tuning algorithm} and \emph{evaluation} 
