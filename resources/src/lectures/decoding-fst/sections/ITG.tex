\frame{
	\frametitle{Ad-hoc distortion limit: expressiveness}
	
	Arbitrarily limit reordering to a fixed-length window \pause
	\begin{itemize}
		\item conveniently (linear complexity), but \pause
		\item what about languages with very different syntax? \\
		e.g. OV vs VO, head-initial vs head-final \pause
		\item can we do better?
	\end{itemize}
	
}

\frame{
	\frametitle{Binary permutations}
	
	Consider a sentence such that $I=4$ \\
	~ let's look at binary trees bracketing this sentence\\ \pause

\scalebox{0.9}{
\textcolor<3-10>{ForestGreen}{
\begin{tikzpicture}[level distance=5pt]
\tikzset{frontier/.style={distance from root=30pt}}
\Tree 
[
	[ 
		[ 1 2 ]
		3
	]
	4
]
\end{tikzpicture}
}
~
\textcolor<11>{ForestGreen}{
\begin{tikzpicture}[level distance=5pt]
\tikzset{frontier/.style={distance from root=30pt}}
\Tree 
[
	[ 
		1
		[ 2 3 ]
	]
	4
]
\end{tikzpicture}
}
~ 
\textcolor<12>{ForestGreen}{
\begin{tikzpicture}[level distance=5pt]
\tikzset{frontier/.style={distance from root=30pt}}
\Tree 
[
	[ 1 2 ]
	[ 3 4 ]
]
\end{tikzpicture}
}
~
\textcolor<13>{ForestGreen}{
\begin{tikzpicture}[level distance=5pt]
\tikzset{frontier/.style={distance from root=30pt}}
\Tree 
[
	1
	[ 
		[ 2 3 ]
		4
	]
]
\end{tikzpicture}
}
~
\textcolor<14>{ForestGreen}{
\begin{tikzpicture}[level distance=5pt]
\tikzset{frontier/.style={distance from root=30pt}}
\Tree 
[
	1
	[ 
		2
		[ 3 4 ]
	]
]
\end{tikzpicture}
}
}

	\only<3->{Binary permutations\\} 
\only<3-10>{
\begin{tabular}{l l}
\only<3-10>{\dbr{\dbr{\dbr{1 2}3}4} & 1 2 3 4 }\\
\only<4-10>{\dbr{\dbr{\ibr{1 2}3}4} & 2 1 3 4 }\\
\only<5-10>{\dbr{\ibr{\dbr{1 2}3}4} & 3 1 2 4 }\\
\only<6-10>{\dbr{\ibr{\ibr{1 2}3}4} & 3 2 1 4 }\\
\only<7-10>{\ibr{\dbr{\dbr{1 2}3}4} & 4 1 2 3 }\\
\only<8-10>{\ibr{\dbr{\ibr{1 2}3}4} & 4 2 1 3 }\\
\only<9-10>{\ibr{\ibr{\dbr{1 2}3}4} & 4 3 1 2 }\\
\only<10-10>{\ibr{\ibr{\ibr{1 2}3}4} & 4 3 2 1 }\\ 
\end{tabular}
}

\only<11>{
\begin{tabular}{l l}
\dbr{\dbr{1\dbr{2 3}}4} & 1 2 3 4 \\
\dbr{\dbr{1\ibr{2 3}}4} & 1 3 2 4 \\
\dbr{\ibr{1\dbr{2 3}}4} & 2 3 1 4 \\
\dbr{\ibr{1\ibr{2 3}}4} & 3 2 1 4 \\
\ibr{\dbr{1\dbr{2 3}}4} & 4 1 2 3 \\
\ibr{\dbr{1\ibr{2 3}}4} & 4 1 3 2 \\
\ibr{\ibr{1\dbr{2 3}}4} & 4 2 3 1 \\
\ibr{\ibr{1\ibr{2 3}}4} & 4 3 2 1 \\ 
\end{tabular}
}

\only<12>{
\begin{tabular}{l l}
\dbr{\dbr{1 2}\dbr{3 4}} & 1 2 3 4\\
\dbr{\ibr{1 2}\dbr{3 4}} & 2 1 3 4\\
\dbr{\ibr{1 2}\ibr{3 4}} & 2 1 4 3 \\
\dbr{\dbr{1 2}\ibr{3 4}} & 1 2 4 3 \\
\ibr{\dbr{1 2}\dbr{3 4}} & 3 4 1 2 \\
\ibr{\ibr{1 2}\dbr{3 4}} & 3 4 2 1 \\
\ibr{\ibr{1 2}\ibr{3 4}} & 4 3 2 1 \\
\ibr{\dbr{1 2}\ibr{3 4}} & 4 3 1 2 \\
\end{tabular}
}

\only<13>{
\begin{tabular}{l l}
\dbr{1\dbr{\dbr{2 3}4}} & 1 2 3 4 \\
\dbr{1\dbr{\ibr{2 3}4}} & 1 3 2 4 \\
\dbr{1\ibr{\dbr{2 3}4}} & 1 4 2 3 \\
\dbr{1\ibr{\ibr{2 3}4}} & 1 4 3 2 \\
\ibr{1\dbr{\dbr{2 3}4}} & 2 3 4 1 \\
\ibr{1\dbr{\ibr{2 3}4}} & 3 2 4 1 \\
\ibr{1\ibr{\dbr{2 3}4}} & 4 2 3 1 \\
\ibr{1\ibr{\ibr{2 3}4}} & 4 3 2 1 \\
\end{tabular}
}

\only<14>{
\begin{tabular}{l l}
\dbr{1\dbr{2\dbr{3 4}}} & 1 2 3 4 \\
\dbr{1\dbr{2\ibr{3 4}}} & 1 2 4 3 \\
\dbr{1\ibr{2\dbr{3 4}}} & 1 3 4 2 \\
\dbr{1\ibr{2\ibr{3 4}}} & 1 4 3 2 \\
\ibr{1\dbr{2\dbr{3 4}}} & 2 3 4 1 \\
\ibr{1\dbr{2\ibr{3 4}}} & 2 4 3 1 \\
\ibr{1\ibr{2\dbr{3 4}}} & 3 4 2 1 \\
\ibr{1\ibr{2\ibr{3 4}}} & 4 3 2 1 \\
\end{tabular}
}

\only<15->{
	\begin{itemize}
		\item<14-> constrains the space of permutations
		\item<15-> crossing constraint (shares a linguistic intuition)\\
		~ 3 1 4 2 \\
		~ 2 4 1 3 \\
	\end{itemize}
}
}


\frame{
	\frametitle{ITGs}
	
	Inversion Transduction Grammars (ITGs) \precite{Wu, 1997} \\
	\pause
	\begin{itemize}
		\item $X \ra X X$ \\
		\textcolor{blue}{direct order} \pause
		\item $X \ra \angbrack{X X}$\\
		\textcolor{red}{inverted order} \pause
		\item $X \ra f/e$, where $(f,e) \in R$\\
		\textcolor{ForestGreen}{bilingual mappings}
	\end{itemize}
	
}

\frame{
	\frametitle{Parsing, intersection and hypergraphs}

\only<1->{	
\begin{textblock*}{80mm}(0.1\textwidth,0.15\textheight)
Source\\
\scalebox{0.6}{
\begin{tikzpicture}[->,>=stealth',shorten >=1pt,auto,node distance=2.8cm,semithick]
%\tikzstyle{every state}=[draw=black,text=black]
%\tikzstyle{every path}=[draw=red,text=red]

\node[initial,state,style={initial text=}] (q0) {$0$};
\node[state] (q1) [right of=q0] {$1$};
\node[state] (q2) [right of=q1] {$2$};
\node[state,accepting] (q3) [right of=q2] {$3$};

\path (q0) edge node {\textit{nosso}} (q1);
\path (q1) edge node {\textit{amigo}} (q2);
\path (q2) edge node {\textit{comum}} (q3);
\end{tikzpicture}
}
\end{textblock*}
}

\only<2->{
\begin{textblock*}{80mm}(0.7\textwidth,0.15\textheight)
\footnotesize
Grammar\\
\begin{tabular}{l l}
$X \ra X X$ & \only<7-10>{\textcolor{blue}{$\Longleftarrow$}}\\
$X \ra \angbrack{X X}$ & \only<11-14>{\textcolor{red}{$\Longleftarrow$}}\\
$X \ra \textit{nosso}$ & \only<4>{\textcolor{ForestGreen}{$\Longleftarrow$}} \\ % /\{\etext{our}, \etext{ours}\}$ \\
$X \ra \textit{amigo}$ & \only<5>{\textcolor{ForestGreen}{$\Longleftarrow$}} \\ % /\{\etext{friend}, \etext{mate}\}$ \\
$X \ra \textit{comum}$ & \only<6>{\textcolor{ForestGreen}{$\Longleftarrow$}}\\ % /\{\etext{ordinary}, \etext{common}, \etext{usual}\}$ \\
\end{tabular}
\end{textblock*}
}

\only<3->{
\begin{textblock*}{80mm}(0.1\textwidth,0.3\textheight)
Intersection \\
\begin{itemize}
	\item \textcolor<4->{ForestGreen}{$_iX_j \ra f$} \\
	{\small s.t. $\angbrack{i,f,j} \in E$}
	\item \textcolor<7->{blue}{$_iX_k \ra {}_iX_j ~ {}_jX_k$} \\
	{\small s.t. $(i,j,k) \in Q^3$}
	\item \textcolor<11->{red}{$_iX_k \ra {}_jX_k ~ {}_iX_j$} \\
	{\small s.t. $(i,j,k) \in Q^3$}
\end{itemize}
\end{textblock*}
}

\begin{textblock*}{80mm}(0.45\textwidth,0.45\textheight)
\scalebox{0.8}{
\begin{tikzpicture}[->,>=stealth',thick]

\coordinate (c01) at (-1,0);
\only<4->{\node[state] (X01) at (1,0) {$_0X_1$};}
\coordinate (c02) at (2,0);
\coordinate (i02) at (2.2,-0.15);
\only<7->{\node[state] (X02) at (4,0) {$_0X_2$};}
\coordinate (c12) at (-1,-2);
\only<5->{\node[state] (X12) at (1,-2) {$_1X_2$};}
\coordinate (c23) at (-1,-4);
\only<6->{\node[state] (X23) at (1,-4) {$_2X_3$};}
\coordinate (c13) at (2,-3);
\coordinate (i13) at (1.95,-3.25);
\only<8->{\node[state] (X13) at (3,-3) {$_1X_3$};}
\coordinate (c03) at (5,0);
\coordinate (i03) at (4.9,-0.2);
\only<9->{\node[state,accepting] (X03) at (7,0) {$_0X_3$};}
\coordinate (c03b) at (4,-1);
\coordinate (i03b) at (4.2,-1.2);


\only<4->{\path[->,color=ForestGreen] (c01) edge node [above] {\textit{nosso}} (X01);}

\only<5->{\path[->,color=ForestGreen] (c12) edge node [above] {\textit{amigo}} (X12);}

\only<6->{\path[->,color=ForestGreen] (c23) edge node [above] {\textit{comum}} (X23);}


\only<7->{
\path[-,color=blue] (X01) edge node {} (c02);
\path[-,color=blue] (X12) edge node {} (c02);
\path[->,color=blue] (c02) edge node {} (X02);
}

\only<8->{
\path[-,color=blue] (X12) edge node {} (c13);
\path[-,color=blue] (X23) edge node {} (c13);
\path[->,color=blue] (c13) edge node {} (X13);
}

\only<9->{
\path[-,color=blue] (X02) edge node {} (c03);
\path[-,color=blue] [bend right=60,looseness=1.6] (X23) edge node {} (c03);
\path[->,color=blue] (c03) edge node {} (X03);
}

\only<10->{
\path[-,color=blue] (X01) edge node {} (c03b);
\path[-,color=blue] (X13) edge node {} (c03b);
\path[->,color=blue] (c03b) edge node {} (X03);
}


\only<11->{
\path[-,color=red] (X01) edge node {} (i02);
\path[-,color=red] (X12) edge node {} (i02);
\path[->,color=red] (i02) edge node {} (X02);
}
\only<12->{
\path[-,color=red] (X12) edge node {} (i13);
\path[-,color=red] (X23) edge node {} (i13);
\path[->,color=red] (i13) edge node {} (X13);
}

\only<13->{
\path[-,color=red] (X02) edge node {} (i03);
\path[-,color=red] [bend right=60,looseness=1.5] (X23) edge node {} (i03);
\path[->,color=red] (i03) edge node {} (X03);
}
\only<14->{
\path[-,color=red] (X01) edge node {} (i03b);
\path[-,color=red] (X13) edge node {} (i03b);
\path[->,color=red] (i03b) edge node {} (X03);
}



\end{tikzpicture}
}
\end{textblock*}

}

\frame{

	\frametitle{ITG solutions}

	\begin{tikzpicture}[->,>=stealth',semithick]%,shorten >=1pt,auto,node distance=1.5cm,semithick]
%\tikzstyle{every state}=[draw=black,text=black]
%\tikzstyle{every path}=[draw=red,text=red]
%\tikzset{dot/.style={circle,fill=#1,inner sep=0,minimum size=4pt}}

\coordinate (c01) at (-1,0);
\node[state] (X01) at (1,0) {$X_{01}$};
\coordinate (c02) at (2,0);
\coordinate (i02) at (2.2,-0.15);
\node[state] (X02) at (4,0) {$X_{02}$};
\coordinate (c12) at (-1,-2);
\node[state] (X12) at (1,-2) {$X_{12}$};
\coordinate (c23) at (-1,-4);
\node[state] (X23) at (1,-4) {$X_{23}$};
\coordinate (c13) at (2,-3);
\coordinate (i13) at (1.95,-3.25);
\node[state] (X13) at (3,-3) {$X_{13}$};
\coordinate (c03) at (5,0);
\coordinate (i03) at (4.9,-0.2);
\node[state,accepting] (X03) at (7,0) {$X_{03}$};
\coordinate (c03b) at (4,-1);
\coordinate (i03b) at (4.2,-1.2);

\coordinate (c01b) at (-1,1);
\coordinate (c12b) at (-1,-1);
\coordinate (c23b) at (-1,-3);
\coordinate (c23c) at (-1,-5);
\coordinate (c23d) at (-1,-6);

\only<1-13>{
\path[->,color=ForestGreen] (c01) edge node [above] {1} (X01);
\path[->,color=ForestGreen] (c12) edge node [above] {2} (X12);
\path[->,color=ForestGreen] (c23) edge node [above] {3} (X23);
}

\path[-,color=blue,densely dotted] (X01) edge node {} (c02);
\path[-,color=blue,densely dotted] (X12) edge node {} (c02);
\path[->,color=blue,densely dotted] (c02) edge node {} (X02);

\path[-,color=blue,densely dotted] (X12) edge node {} (c13);
\path[-,color=blue,densely dotted] (X23) edge node {} (c13);
\path[->,color=blue,densely dotted] (c13) edge node {} (X13);

\path[-,color=blue,densely dotted] (X02) edge node {} (c03);
\path[-,color=blue,densely dotted] [bend right=60,looseness=1.6] (X23) edge node {} (c03);
\path[->,color=blue,densely dotted] (c03) edge node {} (X03);

\path[-,color=blue,densely dotted] (X01) edge node {} (c03b);
\path[-,color=blue,densely dotted] (X13) edge node {} (c03b);
\path[->,color=blue,densely dotted] (c03b) edge node {} (X03);

\path[-,color=red,dashed] (X01) edge node {} (i03b);
\path[-,color=red,dashed] (X13) edge node {} (i03b);
\path[->,color=red,dashed] (i03b) edge node {} (X03);

\path[-,color=red,dashed] (X01) edge node {} (i02);
\path[-,color=red,dashed] (X12) edge node {} (i02);
\path[->,color=red,dashed] (i02) edge node {} (X02);

\path[-,color=red,dashed] (X12) edge node {} (i13);
\path[-,color=red,dashed] (X23) edge node {} (i13);
\path[->,color=red,dashed] (i13) edge node {} (X13);

\path[-,color=red,dashed] (X02) edge node {} (i03);
\path[-,color=red,dashed] [bend right=60,looseness=1.5] (X23) edge node {} (i03);
\path[->,color=red,dashed] (i03) edge node {} (X03);

\only<2-4>{
\path[-,color=blue,thick] (X02) edge node {} (c03);
\path[-,color=blue,thick] [bend right=60,looseness=1.6] (X23) edge node {} (c03);
\path[->,color=blue,thick] (c03) edge node {} (X03);
}
\only<3,6>{
\path[-,color=blue,thick] (X01) edge node {} (c02);
\path[-,color=blue,thick] (X12) edge node {} (c02);
\path[->,color=blue,thick] (c02) edge node {} (X02);
}
\only<4,7>{
\path[-,color=red,thick] (X01) edge node {} (i02);
\path[-,color=red,thick] (X12) edge node {} (i02);
\path[->,color=red,thick] (i02) edge node {} (X02);
}
\only<5-7>{
\path[-,color=red,thick] (X02) edge node {} (i03);
\path[-,color=red,thick] [bend right=60,looseness=1.5] (X23) edge node {} (i03);
\path[->,color=red,thick] (i03) edge node {} (X03);
}
\only<8-10>{
\path[-,color=blue,thick] (X01) edge node {} (c03b);
\path[-,color=blue,thick] (X13) edge node {} (c03b);
\path[->,color=blue,thick] (c03b) edge node {} (X03);
}
\only<9,12>{
\path[-,color=blue,thick] (X12) edge node {} (c13);
\path[-,color=blue,thick] (X23) edge node {} (c13);
\path[->,color=blue,thick] (c13) edge node {} (X13);
}
\only<10,13>{
\path[-,color=red,thick] (X12) edge node {} (i13);
\path[-,color=red,thick] (X23) edge node {} (i13);
\path[->,color=red,thick] (i13) edge node {} (X13);
}
\only<11-13>{
\path[-,color=red,thick] (X01) edge node {} (i03b);
\path[-,color=red,thick] (X13) edge node {} (i03b);
\path[->,color=red,thick] (i03b) edge node {} (X03);
}

\only<14>{\path[->,color=ForestGreen] (c01) edge node [above] {\ftext{nosso}} (X01);}
\only<15->{
\path[->,color=ForestGreen] (c01b) edge node [above] {\etext{our}} (X01);
\path[->,color=ForestGreen] (c01) edge node [above] {\etext{ours}} (X01);
}

\only<14>{\path[->,color=ForestGreen] (c12) edge node [above] {\ftext{amigo}} (X12);}
\only<15->{
\path[->,color=ForestGreen] (c12b) edge node [above] {\etext{friend}} (X12);
\path[->,color=ForestGreen] (c12) edge node [above] {\etext{mate}} (X12);
}

\only<14>{\path[->,color=ForestGreen] (c23) edge node [above] {\ftext{comum}} (X23);}
\only<15->{
\path[->,color=ForestGreen] (c23b) edge node [above] {\etext{usual}} (X23);
\path[->,color=ForestGreen] (c23) edge node [above] {\etext{mutual}} (X23);
\path[->,color=ForestGreen] (c23c) edge node [above] {\etext{common}} (X23);
\path[->,color=ForestGreen] (c23d) edge node [above] {\etext{ordinary}} (X23);
}




\end{tikzpicture}

	\begin{textblock*}{80mm}(0.7\textwidth,0.4\textheight)
	\only<2>{\dbr{~\,...~ 3} \ra \,... 3\\} 
	\only<3->{\dbr{\dbr{1 2}3} \ra 1 2 3\\} 
	\only<4->{\dbr{\ibr{1 2}3} \ra 2 1 3\\} 
	\only<5>{\ibr{~\,...~ 3} \ra 3 \,...\\} 
	\only<6->{\ibr{\dbr{1 2}3} \ra 3 1 2\\} 
	\only<7->{\ibr{\ibr{1 2}3} \ra 3 2 1\\} 
	\only<8>{\dbr{1 ~...~\,} \ra 1 \,...\\}
	\only<9->{\dbr{1\dbr{2 3}} \ra 1 2 3\\}
	\only<10->{\dbr{1\ibr{2 3}} \ra 1 3 2\\}
	\only<11>{\ibr{1 ~...~\,} \ra \,... 1\\}
	\only<12->{\ibr{1\dbr{2 3}} \ra 2 3 1\\}
	\only<13->{\ibr{1\ibr{2 3}} \ra 3 2 1\\}
	\end{textblock*}

}

\frame{
	\frametitle{ITG complexity}
	
	Intersection \\
\begin{itemize}
	\item \textcolor{ForestGreen}{$_iX_j \ra f$} \\
	\item \textcolor{blue}{$_iX_k \ra {}_iX_j ~ {}_jX_k$} \\
	\item \textcolor{red}{$_iX_k \ra {}_jX_k ~ {}_iX_j$} \\
\end{itemize}

	Where
	\begin{itemize}
		\item $(i, j, k) \in \{0, 1, \ldots, I\}^3$
		\item there is a path from $q_i$ to $q_j$ and from $q_j$ to $q_k$ in the input
	\end{itemize}

	Therefore
	\begin{itemize}
		\item $O(I^3)$ nodes
		\item $O(tI^3)$ edges
	\end{itemize}
	
}

\begin{comment}
\frame{
	\frametitle{Lexical evidence}
	
	Hiero style
}

\frame{
	\frametitle{Alternatives}
	
	Syntax augmented \\
	Soft constraints \\
	Latent grammar \\
	
}
\end{comment}

