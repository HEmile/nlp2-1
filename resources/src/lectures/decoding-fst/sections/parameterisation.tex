\section{Parameterisation}

\frame{
	\frametitle{What are we missing?}
	
	We have characterised the set of solutions ``backed'' by our transfer model\\ \pause
	\begin{itemize}
		\item these solutions are unweighted \pause
		\item there is no obvious way to discriminate between them \pause
		\item we cannot make decisions like that
	\end{itemize}
	
	\pause
	
	We are missing a parameterisation of the model
	\begin{itemize}
		\item the scoring function which will guide the decision making process
	\end{itemize}
}

\frame{
	\frametitle{Linear models}
	
	Let's call {\bf derivation}  \pause
	\begin{itemize}
		\item a translation string \pause
		\item along with any latent structure assumed by the transfer model \\
		e.g. phrase segmentation, alignment
	\end{itemize}
	\pause
	
	A linear parameterisation of the model is a function
	$$f(\mdd) = \sum_k \lambda_k H_k(\mdd)$$
	where \mdd is the derivation, and $H_k$ is one of $m$ feature functions\\
	
	~ \pause
	
	It assigns a real-valued score to each and every derivation \\
	
	~ \pause
	
	Think of it as a surrogate for translation quality at decoding time
}

\frame{
	\frametitle{Feature functions}
	
	Independently capture different aspects of the translation, such as
	\begin{itemize}
		\item adequacy
		\begin{itemize}
			\item translation probabilities
			\item confidence on lexical choices
		\end{itemize}
		\item fluency
		\begin{itemize}
			\item LM probabilities
			\item confidence on reodering
		\end{itemize}
	\end{itemize}
	
	
}

\frame{
	\frametitle{Independence assumptions}
	
	Our transfer model makes independence assumptions
	\begin{itemize}
		\item ``translation happens by concatenating isolated rules''
		e.g. flat mappings, hierarchical mappings
	\end{itemize}
	\pause
	
	~
	
	Certain aspects of translation quality comply with such assumptions\looseness=-1
	\begin{itemize}
		\item how likely a certain translation rule is \\
		e.g. relative frequency in a bilingual corpus
	\end{itemize}
	\pause
	
	~
	
	Certain aspects do not comply with such assumptions
	\begin{itemize}
		\item fluency as captured by an $n$-gram LM
	\end{itemize}
	
}

\frame{
	\frametitle{Structural independence}

\begin{textblock*}{110mm}(0.1\textwidth,0.30\textheight)

Suppose a set of derivations (e.g. monotone word mappings)\looseness=-1 \\

	
\scalebox{0.6}{

\begin{tikzpicture}[->,>=stealth',shorten >=1pt,auto,node distance=2.8cm,semithick]
  	\tikzstyle{every state}=[draw=black,text=black]

\node[initial,state,style={initial text=}] (A) {$0$};
\node[state] (B) [right of=A] {$1$};
\node[state] (C) [right of=B] {$2$};
\node[state] (D) [right of=C] {$3$};
\node[state] (E) [right of=D] {$4$};
\node[state,accepting] [right of=E] (F) {$5$};

\only<1>{\path[color=red] (A) edge node {\ftext{um}} (B);}
\only<1-2>{\path[color=red] (B) edge node {\ftext{conto}} (C);}
\only<1-3>{\path[color=red] (C) edge node {\ftext{de}} (D);}
\only<1-4>{\path[color=red] (D) edge node {\ftext{duas}} (E);}
\only<1-5>{\path[color=red] (E) edge node {\ftext{cidades}} (F);}
	
\only<2->{
	\path[color=blue] 
		(A) edge node {\etext{a}} (B)
			edge [bend right] node {\etext{some}} (B)
			edge [bend left] node {\etext{one}} (B);
}
\only<3->{
	\path[color=blue] 
		(B) edge node {\etext{narrative}} (C)
			edge [bend right] node {\etext{tale}} (C)
			edge [bend left] node {\etext{story}} (C)
			edge [bend right=60] node [below] {novella} (C);
}
\only<4->{
	\path[color=blue] 
		(C) edge node {\etext{of}} (D)
			edge [bend right] node {\etext{from}} (D)
			edge [bend left] node {\etext{'s}} (D);
}
\only<5->{
	\path[color=blue] 
		(D) edge [bend left] node {\etext{two}} (E)
			edge [bend right] node {\etext{couple}} (E);
}
\only<6->{
	\path[color=blue] 
		(E) edge node {\etext{cities}} (F)
			edge [bend right] node {\etext{towns}} (F)
			edge [bend left] node {\etext{villages}} (F);
}			
\end{tikzpicture} 

}

\only<7->{
When scoring complies with such ``structural independence''\\
}
\only<8->{\hfill \alert{we work with packed representations \\ \hfill(polynomial in input length)}}

\end{textblock*}
	
}


\frame{
	\frametitle{Structural independence: one extreme}

	If our features are fully compliant with structural independence \\
	~
	
\begin{center}
\begin{math}
\begin{array}{l r}
\only<2->{\mdd = \angbrack{e_1, \ldots, e_l} & \text{sequence of independent steps}\\}
\only<2-5>{& \\}
\only<3->{f(\mdd) = \sum_k \lambda_k H_k(\mdd) & \text{model factorises over features}\\}
\only<2-5>{& \\}
\only<4->{H_k(\mdd) = \sum_{e\in \mdd} h_k(e) & \text{features factorises over steps}\\}
\only<2-5>{& \\}
\only<5->{w(e) = \sum_k \lambda_k h_k(e) & \text{ergo, model factorises over steps}\\}
\end{array}
\end{math}
\end{center}

\only<6->{
\scalebox{0.6}{

\begin{tikzpicture}[->,>=stealth',shorten >=1pt,auto,node distance=3cm,semithick]
  	\tikzstyle{every state}=[draw=black,text=black]
\small
\node[initial,state,style={initial text=}] (A) {$0$};
\node[state] (B) [right of=A] {$1$};
\node[state] (C) [right of=B] {$2$};
\node[state] (D) [right of=C] {$3$};
\node[state] (E) [right of=D] {$4$};
\node[state,accepting] [right of=E] (F) {$5$};

	\path[color=blue] 
		(A) edge node {\etext{a}\only<7->{/$w_2$}} (B)
			edge [bend right] node {\etext{some}\only<7->{/$w_3$}} (B)
			edge [bend left] node {\etext{one}\only<7->{/$w_1$}} (B);


	\path[color=blue] 
		(B) edge node {\etext{narrative}\only<7->{/$w_5$}} (C)
			edge [bend right] node {\etext{tale}\only<7->{/$w_6$}} (C)
			edge [bend left] node {\etext{story}\only<7->{/$w_4$}} (C)
			edge [bend right=60] node [below] {\etext{novella}\only<7->{/$w_7$}} (C);
	\path[color=blue] 
		(C) edge node {\etext{of}\only<7->{/$w_9$}} (D)
			edge [bend right] node {\etext{from}\only<7->{/$w_{10}$}} (D)
			edge [bend left] node {\etext{'s}\only<7->{/$w_8$}} (D);
	\path[color=blue] 
		(D) edge [bend left] node {\etext{two}\only<7->{/$w_{11}$}} (E)
			edge [bend right] node {\etext{couple}\only<7->{/$w_{12}$}} (E);
	\path[color=blue] 
		(E) edge node {\etext{cities}\only<7->{/$w_{14}$}} (F)
			edge [bend right] node {\etext{towns}\only<7->{/$w_{15}$}} (F)
			edge [bend left] node {\etext{villages}\only<7->{/$w_{13}$}} (F);
			
\end{tikzpicture} 
}}

\only<8->{
Efficiently packing $\Rightarrow$ efficient DP-based inference \\
\hfill \alert{(linear with the number of edges)}
}

}

\frame{
	\frametitle{Structural independence: the other extreme}

	Imagine a feature function that requires a complete translation
	\pause
	\begin{itemize}		
		\item unbounded LM \\
		e.g. via suffix arrays or distributed representations
		\item estimated overall translation quality (QE models?)
	\end{itemize}
	\pause
	Steps can no longer be weighted in isolation \pause
	\begin{itemize}
		\item it requires unpacking the representation \pause
		\item making dependencies explicit through the graphical structure
	\end{itemize}
	
	
}

\frame{

	\frametitle{An DP fails...}

\scalebox{0.6}{

\begin{tikzpicture}[->,>=stealth',shorten >=1pt,auto,node distance=2.8cm,semithick]
  	\tikzstyle{every state}=[draw=black,text=black]
  	\tikzstyle{every path}=[draw=blue,text=blue]	

\node[initial,state,style={initial text=}] (q0) {$0$};
\node[state] (q1) [above right of=q0] {$1$};
\node[state] (q2) [right of=q1] {$2$};
\node[state] (q3) [right of=q2] {$3$};
\node[state] (q4) [right of=q3] {$4$};
\node[state,accepting] [right of=q4] (q5) {$5$};

\node[state] (q6) [right of=q0] {$6$};
\node[state] (q7) [right of=q6] {$7$};
\node[state] (q8) [right of=q7] {$8$};
\node[state] (q9) [right of=q8] {$9$};
\node[state,accepting] [right of=q9] (q10) {$10$};

\node[state] (q11) [below right of=q0] {$11$};
\node[state] (q12) [right of=q11] {$12$};
\node[state] (q13) [right of=q12] {$13$};
\node[state] (q14) [right of=q13] {$14$};
\node[state,accepting] [right of=q14] (q15) {$15$};

\node[state] (q16) [below of=q11] {$16$};
\node[state] (q17) [right of=q16] {$17$};

\path
(q0) edge node {\etext{one}} (q1)
(q1) edge node {\etext{story}} (q2)
(q2) edge node {\etext{'s}} (q3)
(q3) edge node {\etext{two}} (q4)
(q4) edge node {\etext{villages}/$w_1$} (q5);

\path
(q0) edge node {\etext{one}} (q6)
(q6) edge node {\etext{story}} (q7)
(q7) edge node {\etext{'s}} (q8)
(q8) edge node {\etext{two}} (q9)
(q9) edge node {\etext{cities}/$w_2$} (q10);

\path
(q0) edge node {\etext{one}} (q11)
(q11) edge node {\etext{story}} (q12)
(q12) edge node {\etext{'s}} (q13)
(q13) edge node {\etext{two}} (q14)
(q14) edge node {\etext{towns}/$w_3$} (q15);

\path
(q0) edge node {\etext{...}} (q16)
(q16) edge node {\etext{...}} (q17);
			
\end{tikzpicture} 
}

~

	Exhaustive enumeration 
	\begin{itemize}
		\item number of edges exponential with input length
		\item \alert{intractable}
	\end{itemize}

}

\frame{
	\frametitle{Not all is lost}
	
	Most features we can reliably estimate \pause
	\begin{itemize}
		\item are rarely sensitive to global context \pause
		\item are quite incremental \pause
		\item compromise between these extreme cases 
	\end{itemize}
	\pause
	
	$n$-gram LMs are good examples
	
	\begin{itemize}
		\item there are up to $|\Delta|^{n-1}$ contexts that must be made explicit \pause
		\item nodes must group derivations sharing the same context \pause
		\item polynomial, though often prohibitive

	\end{itemize}

}

\frame{
	\frametitle{Making context explicit (intuition)}

\scalebox{0.6}{

\begin{tikzpicture}[->,>=stealth',shorten >=1pt,auto,node distance=2.8cm,semithick]
  	\tikzstyle{every state}=[draw=black,text=black]
  	\tikzstyle{every path}=[draw=blue,text=blue]	

\node[initial,state,style={initial text=}] (q0) {$0$};
\only<2>{\node[state] (q1) [right of=q0] {$i$};}
\only<2>{\node[state] (q2) [right of=q1] {$ii$};}
\only<3->{
\node[state] (qOne) [above right of=q0] {$1$};
\node[state] (qA) [right of=q0] {$2$};
\node[state] (qSome) [below right of=q0] {$3$};
}
\only<4->{\node[state] (qTale) at (7,3) {$4$};}
%\node[state] (qTale) [above of=qNarra] {$4$};
%\node[state] (qNovel) [below of=qNarra] {$6$};
%\node[state] (qStory) [below of=qNovel] {$7$};

\only<5->{
\node[state] (qStory) at (12,-2) {$5$};
}

%\node[state,accepting] [right of=q4] (q5) {$5$};

\only<2>{
\path
(q0) edge node {\etext{a}} (q1)
	edge [bend right] node {\etext{some}} (q1)
	edge [bend left] node {\etext{one}} (q1);

\path
(q1) edge node {\etext{narrative}} (q2)
	edge [bend right] node {\etext{tale}} (q2)
	edge [bend left] node {\etext{story}} (q2)
	edge [bend right=60] node [below] {novella} (q2);	
	
}
\only<3->{
\path
(q0) edge node {\etext{a}/$w(\text{a})$} (qA)
	edge [bend right] node {\etext{some}/$w(\text{some})$} (qSome)
	edge [bend left] node {\etext{one}/$w(\text{one})$} (qOne);
}
\only<4->{
\path
(qOne) edge [bend left] node {\etext{tale}/$w(\text{one tale})$} (qTale)
(qA) edge node {\etext{tale}/$w(\text{a tale})$} (qTale)
(qSome) edge [bend right] node [below] {\etext{tale}/$w(\text{some tale})$} (qTale);
}

\only<5->{
\path
(qOne) edge [bend left] node {\etext{story}/$w(\text{one story})$} (qStory)
(qA) edge node {\etext{story}/$w(\text{a story})$} (qStory)
(qSome) edge [bend right] node [below] {\etext{story}/$w(\text{some story})$} (qStory);
}

			
\end{tikzpicture} 
}	
	
}

\frame{
	\frametitle{Recap 4}
	
	\begin{enumerate}
		\item a characterisation the space of solutions \pause
		\item a linear parameterisation of the model \pause
		\item impact of parameterisation on packed representations
	\end{enumerate}
	\pause
	
	What's left? \pause
	\begin{itemize}
		\item more examples of models and impact on representation
		\begin{itemize}
			\item distance-based reordering
			\item lexicalised models
			\item a global feature function
		\end{itemize}
		\item inference algorithms \pause
		\item techniques to make inference feasible for interesting models	
	\end{itemize}
}

