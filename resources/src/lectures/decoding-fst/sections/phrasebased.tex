\frame{
	\frametitle{Recap 2}


%		\pause {\bf {\color{red}but not nearly enough interesting cases!}}
	\begin{enumerate}
		\item 	our first model of translational equivalences assumed {\bf monotonicity} \pause
		\item 	then we incorporated {\bf unconstrained permutations} of the input \pause
		\item 	to avoid NP-completeness, we constrained permutations using a {\bf distortion limit} \pause
		\item 	we can instead constrain permutations using an {\bf ITG} \pause
	\end{enumerate}
	
	\alert{But we still perform translation word-by-word with no insertion or deletion!}

}

\frame{
	\frametitle{1-1 mappings: fail!}
	
	Source: o$_1$ grilo$_2$ \ftext{da}$_3$ lareira$_4$ \\
	Target: the$_1$ cricket$_2$ \itembrack{\etext{\text{on the}}}$_3$ hearth$_4$ \\

}

\frame{
	\frametitle{Insertion and deletion}
	
	Implicitly modelled by moving from words to phrases	 \pause
	\begin{itemize}
		\item a phrase replacement model \pause
		\item operating with an ITG (or with a distortion limit) \pause
		\item with no phrase-insertion or phrase-deletion \pause
		\item constrained to known phrase-to-phrase bilingual mappings \\
	(rule set)
	\end{itemize}
}

\frame{
	\frametitle{Phrase mappings}

	
	Mappings of contiguous sequences of words
	\pause
	\begin{itemize}
		\item learnt directly (e.g. stochastic ITGs) \pause
		\item heuristically extracted from word-aligned data \pause
		\item they might contain unaligned source words (deletions) \pause
		\item they might contain unaligned target words (insertions) \pause
		\item their words need not align monotonically\\
		which gives us a bit of reordering power as well ;) \pause \\
		e.g. \ftext{a loja de antiguidades}/\etext{old curiosity shop}
	\end{itemize}
}

\frame{
	\frametitle{Generalising the rule set (FST)}
	
	
	Rules \\
	\begin{tabular}{l l}
	\ftext{o} & \{\etext{the}, \etext{a}\} \\
	\ftext{grilo} & \{\etext{cricket}, \etext{annoyance}\} \\
	\ftext{da} & \{\etext{on the}, \etext{of}, \etext{from}\} \\
	\ftext{hearth} & \{\etext{lareira}\} \\
	\end{tabular}

	~
	
	\pause
	Using FST
	\begin{itemize}
		\item<2-> each rule can be seen as a transducer
		\item<11-> the union represents the rule set
		\item<13-> standard intersection mechanisms do the rest\\
		\item<14> alternatively, one can perform permutation and translation jointly using a logic program
				
	\end{itemize}
	
\only<3-10>{
\begin{textblock*}{63mm}(0.65\textwidth,0.10\textheight)
\scalebox{1}{	
\begin{tikzpicture}[->,>=stealth',shorten >=1pt,auto,node distance=2.8cm,semithick]
  	\tikzstyle{every state}=[draw=black,text=black]
  	\node[initial,accepting,state,style={initial text=}] (A) {$0$};
  	\only<7>{\node[state] (B) [right of=A] {$1$};}
%  	\only<8>{\node[state] (C) [below of=B] {$2$};}	
%\node[state] (B) [right of=A] {$1$};
%\node[state] (C) [right of=B] {$2$};
%\node[state] (D) [right of=C] {$3$};
%\node[state] (E) [right of=D] {$4$};
%\node[state,accepting] [right of=E] (F) {$5$};
\only<3>{\path (A) edge [loop above] node {\ftext{o}:\etext{the}} (A);}
\only<4>{\path (A) edge [loop above] node {\ftext{o}:\etext{a}} (A);}
\only<5>{\path (A) edge [loop above] node {\ftext{grilo}:\etext{cricket}} (A);}
\only<6>{\path (A) edge [loop above] node {\ftext{grilo}:\etext{annoyance}} (A);}
\only<7>{
\path (A) edge [bend left] node {\ftext{da}:\etext{on}} (B);
\path (B) edge [bend left] node {$\epsilon$:\etext{the}} (A);
}
%\only<8>{
%\path (A) edge [bend left] node {\ftext{da}:$\epsilon$} (B);
%\path (B) edge [bend left] node {$\epsilon$:\etext{on}} (C);
%\path (C) edge [bend left] node {$\epsilon$:\etext{the}} (A);
%}
\only<8>{\path (A) edge [loop above] node {\ftext{da}:\etext{of}} (A);}
\only<9>{\path (A) edge [loop above] node {\ftext{da}:\etext{from}} (A);}
\only<10>{\path (A) edge [loop above] node {\ftext{lareira}:\etext{hearth}} (A);}

\end{tikzpicture} 
}
\end{textblock*}
}

\only<12>{
\begin{textblock*}{63mm}(0.05\textwidth,0.70\textheight)
\scalebox{0.55}{
\begin{tikzpicture}[->,>=stealth',shorten >=1pt,auto,node distance=2.8cm,semithick]
  	\tikzstyle{every state}=[draw=black,text=black]
  	\node[initial,accepting,state,style={initial text=}] (A) {$0$};
  	\node[state] (B) [right of=A] {$1$};
	\node[state] (C) [right of=B] {$2$};
	\node[state] (D) [right of=C] {$3$};
	\node[state] (E) [right of=D] {$4$};
	\node[state] (F) [above of=E] {$5$};
	\node[state] (G) [right of=E] {$6$};
	\node[state] (H) [right of=G] {$7$};
	\node[state,accepting] (I) [right of=H] {$8$};
	
\path (A) edge [loop above] node {\ftext{o}:\etext{the}} (A);
\path (B) edge [loop above] node {\ftext{o}:\etext{a}} (B);
\path (C) edge [loop above] node {\ftext{grilo}:\etext{cricket}} (C);
\path (D) edge [loop above] node {\ftext{grill}:\etext{annoyance}} (D);
\path (E) edge node {\ftext{da}:\etext{on}} (F);
\path (F) edge [bend left] node {$\epsilon$:\etext{the}} (E);
\path (G) edge [loop above] node {\ftext{da}:\etext{of}} (G);
\path (H) edge [loop above] node {\ftext{da}:\etext{from}} (H);
\path (I) edge [loop above] node {\ftext{lareira}:\etext{hearth}} (I);

\path (A) edge node {$\epsilon$:$\epsilon$} (B);
\path (B) edge node {$\epsilon$:$\epsilon$} (C);
\path (C) edge node {$\epsilon$:$\epsilon$} (D);
\path (D) edge node {$\epsilon$:$\epsilon$} (E);
\path (E) edge node {$\epsilon$:$\epsilon$} (G);
\path (G) edge node {$\epsilon$:$\epsilon$} (H);
\path (H) edge node {$\epsilon$:$\epsilon$} (I);
\end{tikzpicture}
}
\end{textblock*}
}

	
}

\frame{
	\frametitle{Phrase permutations' translations with WL$d$}
\begin{columns}
\begin{column}{0.5\textwidth}
	
\newcommand{\biwldadjcond}{
\begin{array}{l }
\textcolor<2>{blue}{i = l}\\
\alert<3>{\oplus_k c_k = \bzero}
\end{array}
}

\newcommand{\biwldnonadjcond}{
\begin{array}{l }
\textcolor<5>{blue}{l < i \leq I} \\
\alert<6>{\delta(i', l) \leq d} \\
\oplus_k c_k = \bzero
\end{array}
}


\begin{math} %\small
 \left. 
  \begin{array}{l}
    \textsc{Item} ~~~~ {\itembrack{[1 .. I +1], \{0,1\}^{d-1}}} \\
    %\\
    \textsc{Goal} ~~~~ {\itembrack{I+1, C}} \\
    %\\  
    \textsc{Axiom} \\ 
	\drule{}{\itembrack{1, 0^{d-1}}}{} \\
    %\\  
    \textsc{Adjacent} \\ 
	\drule{\alert<1>{\itembrack{l, C}} \alert<2>{\angbrack{x_i^{i'} \ra y_j^{j'}}}}{\textcolor<4>{blue}{\itembrack{l + n, C' \ll n}}}{\biwldadjcond} \\ 
	~~ \text{where} \\
	~~~ \alert<4>{C' = \alpha_l^{i+1..i'}(C)} \\
	~~~ \text{$n = \#_1(C') + 1$} \\
%	~~ \alert<4>{\text{\small where $n = \#_1(C') + 1$}} \\
%	~~ \text{\small and $\#_1(C)$ is the number of leading ones in $C$} \\
    \textsc{Non-Adjacent} \\ 
	\drule{\itembrack{l, C} \alert<5>{\angbrack{x_i^{i'} \ra y_j^{j'}}}}{\itembrack{l, \alpha_l^{i..i'}(C)}}{\biwldnonadjcond} \\ 
	
  \end{array} 
\right.
\end{math}


\end{column}
\begin{column}{0.5\textwidth}
	\only<1>{\alert<1>{Permutation window}}
	\only<2>{\textcolor{blue}{Adjacent} \alert{phrase pair}}
	\only<3>{\alert<3>{No overlaps}}
	\only<4>{\alert<4>{update} and \textcolor{blue}{shift}}
	\only<5>{\textcolor{blue}{Non-adjacent} \alert{phrase pair}}
	\only<6>{\alert{distortion limit}}
	\only<7>{
	\begin{itemize}
		\item $O(Id^22^{d-1})$ states \\
		(phrases are limited in length)
		\item $O(tId^22^{d-1})$ transitions
	\end{itemize}
	}
\end{column}
\end{columns}


}

\frame{
	\frametitle{Generalising the rule set (ITG)}
	Simply extend the terminal rules
	\pause
	\begin{itemize}
		\item $X \ra X X$ \\
		direct order \pause
		\item $X \ra \angbrack{X X}$\\
		inverted order \pause
		\item $X \ra r_i$, where $r_i \in R$\\
		\textcolor{ForestGreen}{bilingual mappings}
	\end{itemize}
	\pause
	
	Examples \\
	
	\begin{tabular}{l}
	$X \ra \ftext{o}/\etext{the}$\\
	$X \ra \ftext{grilo}/\etext{cricket}$\\
	$X \ra \ftext{da}/\etext{on the}$\\
	\end{tabular}
	
	\pause
	
	The intersection mechanisms do the rest\\
	\begin{itemize}
		\item $O(I^3)$ nodes (phrases are limited in length)
		\item $O(tI^3)$ edges
	\end{itemize}

}

\frame{
	\frametitle{Recap 3}
	
	We have \pause
	\begin{enumerate}
		\item defined different models of translational equivalence
		\pause
		\begin{itemize}
			\item by translating words or phrases
			\pause
			\item in arbitrary order
			\pause
			\item or according to an ITG
		\end{itemize}
		\pause
		\item efficiently represented the set of translations supported by these models for a given input sentence \pause
		\begin{itemize}
			\item trivially expressed in terms of intersection/composition \pause
			\item a logic program can do the same \\
			(sometimes more convenient, e.g. WL$d$ constraints) 
		\end{itemize}
	\end{enumerate}
}


\frame{
	\frametitle{Remarks}
	
	Phrase-based SMT \precite{Koehn et al, 2003} \pause
	\begin{itemize}
		\item the space of solutions grows linearly with input length and exponentially with the distortion limit
	\end{itemize}
	\pause
	ITG \precite{Wu, 1997} \pause
	\begin{itemize}
		\item the space of solutions is cubic in length \pause
		\item however less efficiently packed, better motivated constraints on reordering
	\end{itemize}
	
}

\frame{
	\frametitle{Remarks (hiero)}	
	
	Hierarchical phrase-based models \precite{Chiang, 2005}
	\pause
	\begin{itemize}
		\item more general SCFG rules (typically up to 2 nonterminals)\\ \pause
		\item weakly equivalent to an ITG\\ 
		(same set of pairs of strings) \pause 
		\item purely lexicalised rules \\ 
		e.g. $X \ra \ftext{loja de antiguidades} / \etext{old curiosity shop}$ \pause
		\item as well as lexicalised recursive rules \\ 
		e.g. $X \ra X_1 \ftext{ de } X_2 \text{ / } X_2 \etext{ 's } X_1$ \pause
		\item no purely unlexicalised rules\footnote{Other than monotone translation with \emph{glue rules}} \\ \pause
		\item same cubic dependency on input length (as ITGs)
		%\item typically estimated heuristically from word-alignment\\
		%(lexical evidence helps here)
	\end{itemize}
	
}
	
