\section{Monotone word replacement models}

\frame{
	\frametitle{Model of translational equivalences}

	Defines the space of possible translations \\
	\begin{itemize}
		\item think of it as a recipe to generate translations \\
		\citep{Lopez:2008:SMT}
	\end{itemize}
	
	\pause
	Example:	\\

	\begin{itemize}
		\item a word replacement model \\ \pause
		\item operates in monotone left-to-right order \pause
		\item with no insertions or deletions \pause
		\item constrained to known word-to-word bilingual mappings \\
		(rule set)
	\end{itemize}
}


\frame{
	\frametitle{Monotone word-by-word translation: solutions}
	\only<1>{
Source: \ftext{um conto de duas cidades}\\

Translation rules\footnote{Unrealistically simple} \\
\begin{tabular}{l l}
\ftext{um} & \{\etext{a}, \etext{some}, \etext{one}\} \\
\ftext{conto} & \{\etext{tale}, \etext{story}, \etext{narrative}, \etext{novella}\} \\
\ftext{de} & \{\etext{of}, \etext{from}, \etext{'s}\} \\
\ftext{duas} & \{\etext{two}, \etext{couple}\} \\
\ftext{cidades} & \{\etext{cities}, \etext{towns}, \etext{villages}\} \\
\end{tabular}
}
\only<2->{
\begin{textblock*}{63mm}(0.6\textwidth,0.15\textheight)
\begin{footnotesize}
\begin{tabular}{l l}
\ftext{um} & \{\etext{a}, \etext{some}, \etext{one}\} \\
\ftext{conto} & \{\etext{tale}, \etext{story}, \etext{narrative}, \etext{novella}\} \\
\ftext{de} & \{\etext{of}, \etext{from}, \etext{'s}\} \\
\ftext{duas} & \{\etext{two}, \etext{couple}\} \\
\ftext{cidades} & \{\etext{cities}, \etext{towns}, \etext{villages}\} \\
\end{tabular}
\end{footnotesize}
\end{textblock*}

\ftext{um conto de duas cidades}\\
}
\only<3->{
\etext{a tale of two cities}\\
}
\only<4->{
\etext{a tale of two \bf towns}\\
}
\only<5->{
\etext{a tale of two \bf villages}\\
}
\only<6->{
\etext{a tale of \bf couple cities}\\
}
\only<7->{
\etext{a tale of couple \bf towns}\\
}
%\only<8->{
%\etext{a tale of couple \bf villages}\\
%}
%\only<9->{
%\etext{a tale \bf from two cities}\\
%}
%\only<10->{
%\etext{a tale from two \bf towns}\\
%}
%\only<11->{
%\etext{a tale from two \bf villages}\\
%}
%\only<12->{
%\etext{a tale from \bf couple cities}\\
%}
%\only<13->{
%\etext{a tale from couple \bf towns}\\
%}
%\only<14->{
%\etext{a tale from couple \bf villages}\\
%}
\only<8->{
\etext{...}\\

\begin{textblock*}{63mm}(0.6\textwidth,0.75\textheight)
This can go very far :(
\end{textblock*}
}

}


\frame{
	\frametitle{Monotone word-by-word translation: complexity}
	
	Say
	\begin{itemize}
		\item the input has $I$ words \\
		\item we know at most $t$ translation options per source word
	\end{itemize}
	
	\pause
	This makes $O(t^I)$ solutions\\
	
	\pause
	Note
	\begin{itemize}
		\item WMT14's shared task: $I=40$ on average
		\item last I checked Moses default was $t = 100$ \\
		\hfill (for a more complex model)
		\item silly monotone word replacement model: $10^{80}$ solutions
	\end{itemize}
	
}


\begin{comment}
\frame{
	\frametitle{Representing discrete sets efficiently}
	\begin{textblock*}{63mm}(0.6\textwidth,0.15\textheight)

\begin{footnotesize}
\begin{tabular}{l l}
\ftext{um} & \{\etext{a}, \etext{some}, \etext{one}\} \\
\ftext{conto} & \{\etext{tale}, \etext{story}, \etext{narrative}, \etext{novella}\} \\
\ftext{de} & \{\etext{of}, \etext{from}, \etext{'s}\} \\
\ftext{duas} & \{\etext{two}, \etext{couple}\} \\
\ftext{cidades} & \{\etext{cities}, \etext{towns}, \etext{villages}\} \\
\end{tabular}
\end{footnotesize}

\end{textblock*}

\begin{textblock*}{63mm}(0.1\textwidth,0.50\textheight)
\scalebox{0.6}{

\begin{tikzpicture}[->,>=stealth',shorten >=1pt,auto,node distance=2.8cm,semithick]
  	\tikzstyle{every state}=[draw=black,text=black]

\node[initial,state,style={initial text=}] (A) {$0$};
\node[state] (B) [right of=A] {$1$};
\node[state] (C) [right of=B] {$2$};
\node[state] (D) [right of=C] {$3$};
\node[state] (E) [right of=D] {$4$};
\node[state,accepting] [right of=E] (F) {$5$};

\only<1>{\path[color=red] (A) edge node {\ftext{um}} (B);}
\only<1-2>{\path[color=red] (B) edge node {\ftext{conto}} (C);}
\only<1-3>{\path[color=red] (C) edge node {\ftext{de}} (D);}
\only<1-4>{\path[color=red] (D) edge node {\ftext{duas}} (E);}
\only<1-5>{\path[color=red] (E) edge node {\ftext{cidades}} (F);}
	
\only<2->{
	\path[color=blue] 
		(A) edge node {\etext{a}} (B)
			edge [bend right] node {\etext{some}} (B)
			edge [bend left] node {\etext{one}} (B);
}
\only<3->{
	\path[color=blue] 
		(B) edge node {\etext{narrative}} (C)
			edge [bend right] node {\etext{tale}} (C)
			edge [bend left] node {\etext{story}} (C)
			edge [bend right=60] node [below] {novella} (C);
}
\only<4->{
	\path[color=blue] 
		(C) edge node {\etext{of}} (D)
			edge [bend right] node {\etext{from}} (D)
			edge [bend left] node {\etext{'s}} (D);
}
\only<5->{
	\path[color=blue] 
		(D) edge [bend left] node {\etext{two}} (E)
			edge [bend right] node {\etext{couple}} (E);
}
\only<6->{
	\path[color=blue] 
		(E) edge node {\etext{cities}} (F)
			edge [bend right] node {\etext{towns}} (F)
			edge [bend left] node {\etext{villages}} (F);
}			
\end{tikzpicture} 

}
\end{textblock*}


\only<7>{
	\begin{textblock*}{63mm}(0.2\textwidth,0.7\textheight)
	$3 \times 4 \times 3 \times 2 \times 3 = 216$ solutions
	\begin{itemize}
		\item $6$ states
		\item $3 + 4 + 3 + 2 + 3 = 15$ transitions
	\end{itemize}
	\end{textblock*}
}

}
\end{comment}

\begin{comment}
\frame{
	\frametitle{Monotone word-by-word translation: expressiveness}
	
	
	Given the ``recipe'' and the set of known mappings \pause
	\begin{enumerate}	
		\item what sentences can we translate? \pause
		\item what translations do we produce? \pause
	\end{enumerate}


	What is the set of sentence pairs defined by this model?
}
\end{comment}

\begin{comment}
\frame{
	\frametitle{Monotone word-by-word translation: expressiveness}

	\only<1>{
Consider this extended set of rules\footnote{Unrealistically simple}\\
\begin{tabular}{l l}
\ftext{um} & \{\etext{a}, \etext{some}, \etext{one}\} \\
\ftext{conto} & \{\etext{tale}, \etext{story}, \etext{narrative}, \etext{novella}\} \\
\ftext{de} & \{\etext{of}, \etext{from}, \etext{'s}\} \\
\ftext{duas} & \{\etext{two}, \etext{couple}\} \\
\ftext{cidades} & \{\etext{cities}, \etext{towns}, \etext{villages}\} \\
\ftext{nosso} & \{\etext{our}, \etext{ours}\} \\
\ftext{amigo} & \{\etext{friend}, \etext{mate}\} \\
\ftext{comum} & \{\etext{ordinary}, \etext{common}, \etext{usual}, \etext{mutual}\} \\

\end{tabular}
}
\only<2->{
\begin{textblock*}{63mm}(0.6\textwidth,0.15\textheight)
\begin{scriptsize}
\begin{tabular}{l l}
\ftext{um} & \{\etext{a}, \etext{some}, \etext{one}\} \\
\ftext{conto} & \{\etext{tale}, \etext{story}, \etext{narrative}, \etext{novella}\} \\
\ftext{de} & \{\etext{of}, \etext{from}, \etext{'s}\} \\
\ftext{duas} & \{\etext{two}, \etext{couple}\} \\
\ftext{cidades} & \{\etext{cities}, \etext{towns}, \etext{villages}\} \\
\ftext{nosso} & \{\etext{our}, \etext{ours}\} \\
\ftext{amigo} & \{\etext{friend}, \etext{mate}\} \\
\ftext{comum} & \{\etext{ordinary}, \etext{common}, \etext{usual}, \etext{mutual}\} \\

\end{tabular}
\end{scriptsize}
\end{textblock*}
}


\begin{textblock*}{63mm}(0.6\textwidth,0.6\textheight)
\only<3->{some of the source sentences...\\}
\only<8->{some of the target sentences...\\}

\only<17->{\vspace{5pt}{\color{OliveGreen}an {\bf infinite} set of source sentences}}
\only<18->{{\color{OliveGreen}each of which has an exponential number of translations}}
\end{textblock*}


\only<3->{\ftext{um conto de duas cidades}\\}
\only<8->{~\etext{a tale of two cities}\\}
\only<9->{~\etext{a tale of two \bf towns}\\}
\only<10->{~\etext{a tale of two \bf villages}\\~\etext{...}\\}
\only<4->{\ftext{{\bf nosso} conto de duas cidades}\\}
\only<11->{~\etext{our story 's couple towns}\\}
\only<12->{~\etext{{\bf ours} story 's couple towns}\\~\etext{...}\\}
\only<5->{\ftext{nosso {\bf amigo} de duas cidades}\\}
\only<13->{~\etext{...}\\}
\only<6->{\ftext{nosso amigo {\bf comum}}\\}
\only<14->{~\etext{...}\\}
\only<7->{\ftext{{\bf um} amigo comum}\\}
\only<15->{~\etext{...}\\}
\only<16->{\ftext{um conto de duas cidades de um amigo comum nosso ...}\\\ftext{...}\\}

%	\begin{textblock*}{63mm}(0.6\textwidth,0.75\textheight)
%	This can go very far...
%	\end{textblock*}

%	\begin{textblock*}{63mm}(0.6\textwidth,0.8\textheight)
%	and it does go {\bf very} far!
%	\end{textblock*}	
}
\end{comment}

\begin{comment}
\frame{
	\frametitle{Compact representation}

	\only<1-2>{Call $R$ the set of rules \\}
\only<2>{
	\begin{center}
	\begin{tikzpicture}[->,>=stealth',shorten >=1pt,auto,node distance=2.8cm,semithick]
	  	\tikzstyle{every state}=[draw=black,text=black]

  		\node[initial,accepting,state,style={initial text=}] (A)                    {$a$};
	
		\path (A) edge [loop above] node {$r_i \in R$} (A);

	\end{tikzpicture}
	\end{center}
}
\only<3>{
	Example \\
	\begin{center}
	\scalebox{0.8}{
	\begin{tikzpicture}[->,>=stealth',shorten >=1pt,auto,node distance=2.8cm,semithick]
	  	\tikzstyle{every state}=[draw=black,text=black]

  		\node[initial,accepting,state,style={initial text=}] (A)                    {$a$};
	
		\path (A) edge [loop above] node {%
			\pbox{4cm}{\scriptsize
				\ftext{um}:\etext{a} \\
				\ftext{um}:\etext{some} \\
				\ftext{um}:\etext{one} \\
				\ftext{conto}:\etext{tale} \\
				\ftext{conto}:\etext{story} \\
				\ftext{conto}:\etext{narrative} \\
				\ftext{conto}:\etext{novella} \\
				\ftext{de}:\etext{of} \\
				\ftext{de}:\etext{from} \\
				\ftext{de}:\etext{'s} \\
				\ftext{duas}:\etext{two} \\
				\ftext{duas}:\etext{couple} \\
				\ftext{cidades}:\etext{cities} \\
				\ftext{cidades}:\etext{towns} \\
				\ftext{cidades}:\etext{villages} \\
				\ftext{nosso}:\etext{our} \\
				\ftext{nosso}:\etext{ours} \\
				\ftext{amigo}:\etext{friend} \\
				\ftext{amigo}:\etext{mate} \\
				\ftext{comum}:\etext{ordinary} \\
				\ftext{comum}:\etext{common} \\
				\ftext{comum}:\etext{usual} \\
				\ftext{comum}:\etext{mutual}
			}						
		} (A);

	\end{tikzpicture}}
	\end{center}
}
}
\end{comment}


\begin{comment}
\frame{
	\frametitle{Space of solutions as intersection}
	
	\begin{center}
	``From the set of sentence pairs compatible with the model, \\
	retain those matching this input''
	\end{center}	
}
\end{comment}

\frame{
	\frametitle{Space of solutions as intersection/composition}
	
	%
% TRANSLATION RULES
%
\begin{textblock*}{63mm}(0.99\textwidth,0.1\textheight)
\scalebox{0.75}{

\begin{tikzpicture}[->,>=stealth',shorten >=1pt,auto,node distance=2.8cm,semithick]
  	\tikzstyle{every state}=[draw=black,text=black]

  		\node[initial,accepting,state,style={initial text=}] (A)                    {$a$};

	\path (A) edge [loop above] node {%
		\pbox{4cm}{\scriptsize
			\ftext{um}:\etext{a} \lefttik{3}{4} \\
			\ftext{um}:\etext{some} \lefttik{4}{5} \\
			\ftext{um}:\etext{one} \lefttik{5}{6} \\
			\ftext{conto}:\etext{tale} \lefttik{6}{7} \\
			\ftext{conto}:\etext{story} \lefttik{7}{8}\\
			\ftext{conto}:\etext{narrative} \lefttik{8}{9}\\
			\ftext{conto}:\etext{novella}\lefttik{9}{10}\\
			\ftext{de}:\etext{of} \lefttik{10}{11}\\
			\ftext{de}:\etext{from} \lefttik{11}{12}\\
			\ftext{de}:\etext{'s} \lefttik{12}{13}\\
			\ftext{duas}:\etext{two} \lefttik{13}{14}\\
			\ftext{duas}:\etext{couple} \lefttik{14}{15}\\
			\ftext{cidades}:\etext{cities} \lefttik{15}{16}\\
			\ftext{cidades}:\etext{towns} \lefttik{16}{17}\\
			\ftext{cidades}:\etext{villages} \lefttik{17}{18}
		}						
	} (A);

\end{tikzpicture}
} % scale-box
\end{textblock*}

%
% INPUT
%
\begin{textblock*}{63mm}(0.01\textwidth,0.2\textheight)
\scalebox{0.6}{	
\begin{tikzpicture}[->,>=stealth',shorten >=1pt,auto,node distance=2.8cm,semithick]
  	\tikzstyle{every state}=[draw=black,text=black]
  	\node[initial,state,style={initial text=}] (A) {$0$};
\node[state] (B) [right of=A] {$1$};
\node[state] (C) [right of=B] {$2$};
\node[state] (D) [right of=C] {$3$};
\node[state] (E) [right of=D] {$4$};
\node[state,accepting] [right of=E] (F) {$5$};
\path[color=red] 
	(A) edge node {\ftext{um}\only<3-5>{$^\star$}} (B)
	(B) edge node {\ftext{conto}\only<6-9>{$^\star$}} (C)
	(C) edge node {\ftext{de}\only<10-12>{$^\star$}} (D)
	(D) edge node {\ftext{duas}\only<13-14>{$^\star$}} (E)
	(E) edge node {\ftext{cidades}\only<15-17>{$^\star$}} (F);
\end{tikzpicture} 
}
\end{textblock*}

%
% INTERSECTION
%
\begin{textblock*}{63mm}(0.01\textwidth,0.45\textheight)
\scalebox{0.65}{

\begin{tikzpicture}[->,>=stealth',shorten >=1pt,auto,node distance=2.8cm,semithick]
  	\tikzstyle{every state}=[draw=black,text=black]
\tikzstyle{every path}=[draw=blue,text=blue]

\only<2->{
\node[initial,state,style={initial text=}] (A) {$0,a$};
\node[state] (B) [right of=A] {$1,a$};
\node[state] (C) [right of=B] {$2,a$};
\node[state] (D) [right of=C] {$3,a$};
\node[state] (E) [right of=D] {$4,a$};
\node[state,accepting] [right of=E] (F) {$5,a$};
}

	
\only<3->{\path (A) edge [bend left] node {\etext{a}} (B);}
\only<4->{\path (A) edge node {\etext{some}} (B);}
\only<5->{\path (A) edge [bend right] node {\etext{one}} (B);}

\only<6->{\path (B) edge [bend left] node {\etext{tale}} (C);}
\only<7->{\path (B) edge node {\etext{story}} (C);}
\only<8->{\path (B) edge [bend right] node {\etext{narrative}} (C);}
\only<9->{\path (B) edge [bend right=60] node [below] {novella} (C);}

\only<10->{\path (C) edge [bend left] node {\etext{of}} (D);}
\only<11->{\path (C) edge node {\etext{from}} (D);}
\only<12->{\path (C) edge [bend right] node {\etext{'s}} (D);}

\only<13->{\path (D) edge [bend left] node {\etext{two}} (E);}
\only<14->{\path (D) edge [bend right] node {\etext{couple}} (E);}

\only<15->{\path(E) edge [bend left] node {\etext{cities}} (F);}
\only<16->{\path(E) edge node {\etext{towns}} (F);}
\only<17->{\path(E) edge [bend right] node {\etext{villages}} (F);}			
\end{tikzpicture} 

}
\end{textblock*}	
	
	\only<18>{
	\begin{textblock*}{63mm}(0.2\textwidth,0.75\textheight)
	$3 \times 4 \times 3 \times 2 \times 3 = 216$ solutions
	\begin{itemize}
		\item $6$ states
		\item $3 + 4 + 3 + 2 + 3 = 15$ transitions
	\end{itemize}
	\end{textblock*}
	}

}

\frame{
	\frametitle{Packing solutions with finite-state automata}
	Same $O(t^I)$ solutions using
	\begin{itemize}
		\item $O(I)$ states
		\item $O(tI)$ transitions
	\end{itemize}
}



\frame{
	\frametitle{Recap 1}
	
	\pause
	Model of translational equivalences 
	\begin{itemize}
		\item defines the space of possible sentence pairs
		\item conveniently decomposes into smaller bilingual mappings
	\end{itemize}
	\pause
	Monotone word replacement model
	\begin{itemize}
		\item easy to represent using finite-state transducers \pause
		\item set of translations given by composition \pause
		\item exponential number of solutions in linear space \pause
		\item translates infinitely many sentences \\
		\pause {\bf {\color{red}but not nearly enough interesting cases!}}
	\end{itemize}
}


\frame{
	\frametitle{Monotone word-by-word translation: fail!}
	
		
\begin{textblock*}{63mm}(0.1\textwidth,0.25\textheight)
\begin{small}
\begin{tabular}{l l}
\ftext{nosso} & \{\etext{our}, \etext{ours}\} \\
\ftext{amigo} & \{\etext{friend}, \etext{mate}\}\\
\ftext{comum} & \{\etext{ordinary}, \etext{common}, \etext{usual}, \etext{mutual}\} \\
\end{tabular}
\end{small}
\end{textblock*}

\begin{textblock*}{90mm}(0.1\textwidth,0.50\textheight)
\scalebox{0.7}{

\begin{tikzpicture}[->,>=stealth',shorten >=1pt,auto,node distance=2.8cm,semithick]
  	\tikzstyle{every state}=[draw=black,text=black]

\node[initial,state,style={initial text=}] (A) {$0$};
\node[state] (B) [right of=A] {$1$};
\node[state] (C) [right of=B] {$2$};
\node[state,accepting] (D) [right of=C] {$3$};

\only<1>{\path[color=red] (A) edge node {\ftext{nosso}} (B);}
\only<1-2>{\path[color=red] (B) edge node {\ftext{amigo}} (C);}
\only<1-3>{\path[color=red] (C) edge node {\ftext{comum}} (D);}
	
\only<2->{
	\path[color=blue] 
		(A) edge [bend right] node {\etext{our}} (B)
			edge [bend left] node {\etext{ours}} (B);
}
\only<3->{
	\path[color=blue] 
		(B) edge [bend right] node {\etext{mate}} (C)
			edge [bend left] node {\etext{friend}} (C);
}
\only<4->{
	\path[color=blue] 
		(C) edge node {\etext{common}} (D)
			edge [bend right] node {\etext{usual}} (D)
			edge [bend left] node {\etext{ordinary}} (D)
			edge [bend right=60] node [below] {mutual} (D);
}
\end{tikzpicture} 
}


\only<5->{
	\vspace{5pt}
	We simply cannot obtain a correct translation \\
	\begin{center}\color{OliveGreen}
	our mutual friend
	\end{center}
}

\end{textblock*}
}
