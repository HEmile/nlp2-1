\only<1-2>{Call $R$ the set of rules \\}
\only<2>{
	\begin{center}
	\begin{tikzpicture}[->,>=stealth',shorten >=1pt,auto,node distance=2.8cm,semithick]
	  	\tikzstyle{every state}=[draw=black,text=black]

  		\node[initial,accepting,state,style={initial text=}] (A)                    {$a$};
	
		\path (A) edge [loop above] node {$r_i \in R$} (A);

	\end{tikzpicture}
	\end{center}
}
\only<3>{
	Example \\
	\begin{center}
	\scalebox{0.8}{
	\begin{tikzpicture}[->,>=stealth',shorten >=1pt,auto,node distance=2.8cm,semithick]
	  	\tikzstyle{every state}=[draw=black,text=black]

  		\node[initial,accepting,state,style={initial text=}] (A)                    {$a$};
	
		\path (A) edge [loop above] node {%
			\pbox{4cm}{\scriptsize
				\ftext{um}:\etext{a} \\
				\ftext{um}:\etext{some} \\
				\ftext{um}:\etext{one} \\
				\ftext{conto}:\etext{tale} \\
				\ftext{conto}:\etext{story} \\
				\ftext{conto}:\etext{narrative} \\
				\ftext{conto}:\etext{novella} \\
				\ftext{de}:\etext{of} \\
				\ftext{de}:\etext{from} \\
				\ftext{de}:\etext{'s} \\
				\ftext{duas}:\etext{two} \\
				\ftext{duas}:\etext{couple} \\
				\ftext{cidades}:\etext{cities} \\
				\ftext{cidades}:\etext{towns} \\
				\ftext{cidades}:\etext{villages} \\
				\ftext{nosso}:\etext{our} \\
				\ftext{nosso}:\etext{ours} \\
				\ftext{amigo}:\etext{friend} \\
				\ftext{amigo}:\etext{mate} \\
				\ftext{comum}:\etext{ordinary} \\
				\ftext{comum}:\etext{common} \\
				\ftext{comum}:\etext{usual} \\
				\ftext{comum}:\etext{mutual}
			}						
		} (A);

	\end{tikzpicture}}
	\end{center}
}