
\newcommand{\argmax}[1]{\underset{#1}{\mathrm{argmax}}\ }

\newcommand{\highlight}[1]{%
  \colorbox{red!50}{$\displaystyle#1$}}

\newlength{\overwritelength}
\newlength{\minimumoverwritelength}
\setlength{\minimumoverwritelength}{1cm}
\newcommand{\overwrite}[3][red]{%
  \settowidth{\overwritelength}{$#2$}%
  \ifdim\overwritelength<\minimumoverwritelength%
    \setlength{\overwritelength}{\minimumoverwritelength}\fi%
  \stackrel
    {%
      \begin{minipage}{\overwritelength}%
        \color{#1}\centering\small #3\\%
        \rule{1pt}{9pt}%
      \end{minipage}}
    {\colorbox{#1!50}{\color{black}$\displaystyle#2$}}}



\frame{\frametitle{Going from Generative to Discriminative models}
Start with generative noisy channel model:
\begin{align}
t^* & = \argmax{t \in T(s)} p(t|s) 
\onslide<2->{ = \argmax{t \in T(s)}  \frac{p(s|t)p(t)}{p(s)}}
\onslide<3->{ = \argmax{t \in T(s)}  p(s|t)p(t)  \nonumber \\}
\onslide<4->{& = \argmax{t \in T(s)}  \log p(s|t) + \log p(t)  \nonumber \\}
\onslide<5->{& = \argmax{t \in T(s)}  \begin{bmatrix} 1 & 1 \end{bmatrix} \begin{bmatrix}\log p(s|t) \\ \log p(t)\end{bmatrix} \nonumber \\}
\onslide<6->{& = \highlight{\argmax{t \in T(s)}  \mathbf{\lambda}^T \mathbf{h}(s,t)} \text{\ \ \ \ \ \ end with linear discriminative model} \nonumber}
\end{align}
% \onslide<6->{End with linear discriminative model}
\onslide<7->{
Why would we want to do this?
\onslide<8->{
\begin{itemize}
\item We can add more indicators (features) of good translation
\onslide<9->{\item We can give different weight to different features
\onslide<10->{\item And all this done in a way to directly optimize desired metric}
\end{itemize}
}
}
}
\onslide<11->{
Disadvantage? Losing probabilistic interpretation
}

}
