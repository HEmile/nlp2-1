
%\documentclass[xcolor=pdftex,dvipsnames,table]{beamer}
\documentclass[xcolor=pdftex,dvipsnames,table]{beamer}

\usetheme{Amsterdam}

\usefonttheme[onlymath]{serif}
\setbeamertemplate{navigation symbols}{}
\setbeamertemplate{footline}[frame number]


\usepackage{lmodern}
\usepackage{comment}
\usepackage{natbib}
\usepackage{graphicx}
%\usepackage{pgf}
\usepackage{pbox}
\usepackage{pifont}
\usepackage{alltt}
\usepackage{verbatim}
\usepackage{multirow}
\usepackage{xspace}
\usepackage{cases}
\usepackage{geometry}
%\usepackage{tikz}
%\usetikzlibrary{arrows,automata,positioning}
%\usepackage[absolute,overlay]{textpos}
\usepackage[normalem]{ulem}
%\usepackage{tikz-qtree}
\usepackage{pbox}
%\usepackage{ragged2e}
%\usepackage{pifont}

%\usepackage[all]{xy}

\usepackage{latexsym}
\usepackage{amsmath}
\usepackage{amssymb}
\usepackage{xfrac}

\usepackage{notation}
\usepackage{variables}
\usepackage{drawing}
%\usepackage{xparse}
%\usepackage{tikz}
%\usetikzlibrary{calc}



\newcommand{\dquote}[1]{``{#1}''}
\newcommand{\squote}[1]{`{#1}'}
\newcommand{\pgivenbf}[3]{${#1}(\mathbf{#2} | \mathbf{#3})$}
\newcommand{\pgiven}[3]{${#1}({#2} | {#3})$}
\newcommand{\pofbf}[2]{${#1}(\mathbf{#2})$}
\newcommand{\pof}[2]{${#1}({#2})$}
\newcommand{\indice}[1]{$_{#1}$}
\newcommand{\cgray}[1]{\textcolor{gray}{{#1}}}
\newcommand{\cblue}[1]{\textcolor{blue}{{#1}}}
\newcommand{\cred}[1]{\textcolor{red}{{#1}}}
\newcommand{\coran}[1]{\textcolor{Orange}{{#1}}}
\newcommand{\cgreen}[1]{\textcolor{Green}{{#1}}}
\newcommand{\cellbl}{\cellcolor{black}}
\newcommand{\cellblue}{\cellcolor{blue}}
\newcommand{\cellgreen}{\cellcolor{green}}
\newcommand{\cellr}{\cellcolor{red}}
\newcommand{\cellg}{\cellcolor{gray}}
\newcommand{\celldg}{\cellcolor{darkgray}}
\newcommand{\lra}{$\leftrightarrow$}
\newcommand{\WX}{\textcolor{white}{X}}
\newcommand{\WDot}{\textcolor{white}{$\cdot$}}
\newcommand{\WN}[1]{\textcolor{white}{#1}}
\newcommand{\vtext}[1]{\begin{sideways}#1\end{sideways}}
\newcommand{\phr}[1]{$\overset{\_}{#1}$}
\newcommand{\phs}{\overset{\_}{s}}
\newcommand{\pht}{\overset{\_}{t}}
\newcommand{\hypoe}[2]{\fbox{$\overset{\text{#1}}{#2}$}}
\newcommand{\hypot}[2]{\fbox{$\overset{\text{#1}}{\text{#2}}$}}




% declares a document
\begin{document}



	%\title{Employee's social media use}
	%\title{Social media use by employees}
	\title{Phrase-based SMT}
	%\subtitle{for unsupervised language learning}

	\author{Sophie Arnoult}
	\institute[UvA]{
		%\inst{1}
		Universiteit van Amsterdam\\
	}

	\date{\today}
	
	% Title page
	{\setbeamertemplate{footline}{}
	\begin{frame}[plain]
		\titlepage
	\end{frame}
	}


	% Table of contents	
	%\frame[allowframebreaks]{
	{\setbeamertemplate{footline}{}
	\begin{frame}
		\frametitle{Content}
		\tableofcontents
	\end{frame}
	}



	% trick to start counting from the table of contents
	\setcounter{framenumber}{0}


	% SLIDES
	\section{Word-based SMT}
\subsection{IBM models 1 and 2}
\frame{
    \frametitle{The Noisy-Channel approach}
	
	Bayes rule 
	
	$$P(E|F) = \frac{P(E)P(F|E)}{P(F)}$$
	
	Inference
	
	$$\hat{E} = \argmax_E P(E)P(F|E)$$
	
	Estimation
	
	\begin{itemize}
		\item $P(E)$ $n$-gram LM
		\item $P(F|E)$ ...
	\end{itemize}
	
}

\frame{
    \frametitle{The IBM models}

		$$P(F|E) = \sum_A P(A,F|E) $$
	\begin{center}

    	 \includegraphics[width=0.8\textwidth]{"img/wilker-align"}
	\end{center}
}


\frame{
    \frametitle{Models 1 and 2}
	\begin{align*}
	P(F, A | E) &= P(m| E) \prod_{j=1}^m P(a_j,f_j | a_1^{j-1}, f^{j-1}_1,m, E) \\
		&= P(m| E) \prod_{j=1}^m P(a_j|a_1^{j-1}, f_1^{j-1}, m, E) P (f_j | a_1^{j}, f^{j-1}_1,m, E) 
	\end{align*}
	\pause
	\begin{itemize}
	\item lexical translation 
		$P(f_j | a_1^{j}, f^{j-1}_1,m, E) = t(f_j|e_i)$
	\item alignment $P(a_j| \dots)$
		\begin{itemize}
		\item IBM1: $\sim unif(l+1)$
		\item IBM2: $= a(i|j,m,l)$
		\item HMM: $= a(i|a_{j-1},l)$
		\end{itemize}
	\end{itemize}
}

\frame{
    \frametitle{Decoding with models 1 \& 2?}
	\begin{center}
    	 \includegraphics[width=0.8\textwidth]{"img/wilker-align"}
	\end{center}
	
	~
	\pause
	how to explain insertions on the English side?
}

\subsection{Model 3}
\frame{
    \frametitle{Modelling word fertility}

	\begin{itemize}
	\item {\em fertility}: number of words generated by an English words
	\item Generative story
	\begin{itemize}
	\item choose fertility for $e_i$
	\item choose French words generated for each $e_i$
	\item reorder French words
	\end{itemize}
	\pause
	\item parameters: fertility, translation, distortion, null-word
	\pause
	\item inference is intractable: \\
	~ E step in neighbourhood of Viterbi alignment
	\end{itemize}
}

\frame{
    \frametitle{Generative story}
	\begin{center}
       \only<1>{
                \includegraphics[width=0.8\textwidth]{"img/ibm3"}
        }
        \only<2>{
                \includegraphics[width=0.8\textwidth]{"img/ibm3-1"}
        }
        \only<3>{
                \includegraphics[width=0.8\textwidth]{"img/ibm3-2"}
        }
        \only<4>{
                \includegraphics[width=0.8\textwidth]{"img/ibm3-3"}
        }
        \only<5>{
                \includegraphics[width=0.8\textwidth]{"img/ibm3-4"}
        }
	\end{center}

}

\frame{
    \frametitle{Conclusion}
		
	\begin{center}
       \includegraphics[width=0.5\textwidth]{"img/wilker-ibm-table"}
	\end{center}

	~

	\begin{itemize}
	\item IBM models 1 and 2 are too weak  for decoding 
	\item decoding is NP-complete (for phrase-based models too) 
	\item asymmetry is unsatisfactory from linguistic perspective
	\end{itemize}
	
	
}


	\section{Phrase-based SMT}
\subsection{Motivation}
\frame{
    \frametitle{From word-based to phrase-based SMT}
Capturing non-compositional translation equivalents
	\begin{itemize}
	\item multi-word expressions
		\begin{itemize}
		\item (Fr) ``est-ce que'' $\leftrightarrow$ ``do/did''
		\item ``kick the bucket'' $\leftrightarrow$ ``die''
		\end{itemize}
	\pause
	\item morphology / inflection
		\begin{itemize}
		\item very limited in English
		\item verb inflection and noun agreement in Romance languages
		\pause
		\item ``est-ce que tu voulais'' $\leftrightarrow$ ``did you want''\\
			({\em ? you want-P{\scriptsize ast}-you} $\leftrightarrow$ {\em ?-P{\scriptsize ast} you want})
		\item ``tu as gagn\'e / gagnais '' $\not\leftrightarrow$ ``you won / have won'' (aspect)\\
		\end{itemize}
	\pause
	\item local reorderings \\
		\begin{itemize}
		\item ``un homme grand'' $\leftrightarrow$ ``a tall man'' \\ 
		\item ``un grand homme''$\leftrightarrow$ ``a great man''
		\end{itemize}
	\end{itemize}

}

\frame{
    \frametitle{Example}

	\includegraphics[scale=0.5]{"img/PB extraction 2"}
	
}

\frame{
    \frametitle{Generative story}
	A new hidden variable: segmentation $S$

	 ~

	One possible story
	\begin{align*}
		P(F|E) 
			&= \sum_S \sum_A P(S,A,F|E) \\
			&= \sum_S \sum_A P(S|E) \times P(A|S,E) \times P(F|A,S, E) \\
			%&= \sum_A \prod_k t(\bar{f}_k|\bar{e}_k) d(\text{start}_k - \text{end}_{k-1} - 1)
	\end{align*}

	\pause
	\begin{center}
	\begin{tabular}{l  r}
		I$_1$ have$_2$ black$_3$ eyes$_4$ & input\\ \pause
		{[\textcolor{blue}{I$_1$ have$_2$}] [\textcolor{Green}{black$_3$ }] [\textcolor{red}{eyes$_4$}]} & segmentation\\ \pause
		{[\textcolor{blue}{I$_1$ have$_2$}]$_1$ [\textcolor{red}{eyes$_4$}]$_3$ [\textcolor{Green}{black$_3$ }]$_2$} & ordering\\		\pause
		{[\textcolor{blue}{J' ai}]$_1$ [\textcolor{red}{les yeux}]$_3$ [\textcolor{Green}{noirs}]$_2$} & translation \\			
	\end{tabular}
	\end{center}

}

\subsection{Generative Modelling}
\frame{
    \frametitle{(MLE) inference in phrase-based models}
	\begin{itemize}
	\item \citep{Marcu+2002:JPBSMT}
		\begin{itemize}
		\item hidden segmentation and alignment
		\item uniform segmentation, infer distortion and translation probabilities 
		\pause
		\item approximate inference, overfitting
		\end{itemize}
	\pause
	\item \citep{DeNero+2006:heuristics}\\
		\begin{itemize}
		\item comparable to \citep{Marcu+2002:JPBSMT}, with observed (word) alignments
		\pause
		\item approximate inference, overfitting
		\end{itemize}
	\pause
	\item \citep{Koehn+2003:pbsmt}\\
		\begin{itemize}
		\item observed (word) alignment and phrase pairs (\alert{not segmentations!})
		\item parametric distortion and heuristic translation estimates  
		\end{itemize}
	\pause
	\item \citep{Mylonakis+2008:itgprior}
		\begin{itemize}
		\item observed (word) alignment and phrase pairs
		\item ITG-based segmentation, infer translation probabilities  
		\end{itemize}
	\end{itemize}

}


\subsection{Discriminative modelling}
\frame{
    \frametitle{The Alignment-Template Approach}

	\citep{Och+2000:atm} laid out the fundations for \citep{Koehn+2003:pbsmt}

	~
\begin{center}	
\includegraphics[scale=0.4]{"img/och99-atm"}
\end{center}
	~

	Alignment template: (class) phrase pair \& internal alignment
	
}

\frame{
    \frametitle{Model}
        \begin{align*}
                \text{score}(E,S, A|F) 
                        &= \theta^\top h(F, E, A, S)
        \end{align*}

	~
	\pause
	features include
	\begin{itemize}
	\item language model
	\item alignment (distortion)
	\item translation
	\end{itemize}
	~
	\pause
	independence assumptions
	\begin{itemize}
	\item $h_A(F,E,A,S)= \log \prod_k p(a_k|F,E,A,S)$
	\item $h_F(F,E,A,S)= \log \prod_k p(\bar{f}_k|F,E,A,S)$
	\end{itemize}	
}

\frame{
    \frametitle{Alignment symmetrization}
\begin{center}
\includegraphics[scale=0.45]{"img/symal"}
\end{center}
}

\frame{
    \frametitle{Alignment consistency}

				Let $(\bar{f},\bar{e})$ be a phrase pair\\
				Let $A$ be an alignment matrix\\
				\pause
				\begin{block}{$(\bar{f},\bar{e})$ consistent with $A$ if, and only if:}
					\begin{itemize}
						\pause
						\item Words in $\bar{f}$, if aligned, align only with words in $\bar{e}$\\
						\pause	
						\begin{tiny}
						\begin{columns}
						\begin{column}{1cm}
						\begin{tabular}{|p{0.1cm}|p{0.1cm}|p{0.1cm}|}
							\multicolumn{3}{c}{\cblue{C}} \\ \hline 
							\cellg $\bullet$ & \cellg & \cellg \\ \hline
							\cellg & \cellg $\bullet$ & \cellg $\bullet$ \\ \hline
							 &  & \\ \hline
						\end{tabular}
						\end{column}
						\begin{column}{1cm}
						\begin{tabular}{|p{0.1cm}|p{0.1cm}|p{0.1cm}|}
							\multicolumn{3}{c}{\cblue{C}} \\ \hline
							\cellg $\bullet$ & \cellg & \cellg \\ \hline
							\cellg & \cellg $\bullet$ & \cellg $\bullet$ \\ \hline
							\cellg & \cellg & \cellg \\ \hline
						\end{tabular}
						\end{column}
						\begin{column}{1cm}
						\begin{tabular}{|p{0.1cm}|p{0.1cm}|p{0.1cm}|}
							\multicolumn{3}{c}{\cred{I}} \\ \hline
							\cellg $\bullet$ & \cellg & \\ \hline
							\cellg & \cellg $\bullet$ & \textcolor{red}{$\bullet$} \\ \hline
							 &  & \\ \hline
						\end{tabular}
						\end{column}
						\end{columns}
						\end{tiny}
						
						\pause
						\item Words in $\bar{e}$, if aligned, align only with words in $\bar{f}$\\
						\pause
						\begin{tiny}
						\begin{columns}
						\begin{column}{1cm}
						\begin{tabular}{|p{0.1cm}|p{0.1cm}|p{0.1cm}|}
							\multicolumn{3}{c}{\cblue{C}} \\ \hline
							\cellg $\bullet$ & \cellg & \\ \hline
							\cellg & \cellg $\bullet$ & \\ \hline
							\cellg & \cellg $\bullet$ & \\ \hline
						\end{tabular}
						\end{column}
						\begin{column}{1cm}
						\begin{tabular}{|p{0.1cm}|p{0.1cm}|p{0.1cm}|}
							\multicolumn{3}{c}{\cblue{C}} \\ \hline
							\cellg $\bullet$ & \cellg & \cellg \\ \hline
							\cellg & \cellg $\bullet$ & \cellg \\ \hline
							\cellg & \cellg $\bullet$ & \cellg \\ \hline
						\end{tabular}
						\end{column}
						\begin{column}{1cm}
						\begin{tabular}{|p{0.1cm}|p{0.1cm}|p{0.1cm}|}
							\multicolumn{3}{c}{\cred{I}} \\ \hline
							\cellg $\bullet$ & \cellg & \\ \hline
							\cellg & \cellg $\bullet$ & \\ \hline
							 & \textcolor{red}{$\bullet$} & \\ \hline
						\end{tabular}
						\end{column}
						\end{columns}
						\end{tiny}
								
						\pause			
						\item $(\bar{f},\bar{e})$ must contain at least one alignment point\\
						\pause
						\begin{tiny}
						\begin{columns}
						\begin{column}{1cm}
						\begin{tabular}{|p{0.1cm}|p{0.1cm}|p{0.1cm}|}
							\multicolumn{3}{c}{\cblue{C}} \\ \hline
							\cellg $\bullet$ & \cellg & \cellg \\ \hline
							\cellg  &\cellg $\bullet$ & \cellg \\ \hline
							\cellg  & \cellg & \cellg \\ \hline
						\end{tabular}
						\end{column}
						\begin{column}{1cm}
						\begin{tabular}{|p{0.1cm}|p{0.1cm}|p{0.1cm}|}
							\multicolumn{3}{c}{\cblue{C}} \\ \hline
							\cellg $\bullet$ & & \\ \hline
							  & \cellg $\bullet$ & \cellg \\ \hline
							 & \cellg & \cellg \\ \hline
						\end{tabular}
						\end{column}
						\begin{column}{1cm}
						\begin{tabular}{|p{0.1cm}|p{0.1cm}|p{0.1cm}|}
							\multicolumn{3}{c}{\cred{I}} \\ \hline
							$\bullet$ &  & \\ \hline
							 & $\bullet$ & \\ \hline
							 &  & \cellg \\ \hline
						\end{tabular}
						\end{column}
						\end{columns}
						\end{tiny}						
						
					\end{itemize}
				\end{block}
	
}

\frame{
    \frametitle{Phrase extraction}
	
	%\citet{Koehn+2003:PBSMT}
	\only<1>{
		\includegraphics[scale=0.5]{"img/PB extraction 0"}
	}
	\only<2>{
		\includegraphics[scale=0.5]{"img/PB extraction 1"}
	}
	\only<3>{
		\includegraphics[scale=0.5]{"img/PB extraction 1b"}
	}
	\only<4>{
		\includegraphics[scale=0.5]{"img/PB extraction 2"}
	}
	\only<5>{
		\includegraphics[scale=0.5]{"img/PB extraction 2b"}
	}
	\only<6>{
		\includegraphics[scale=0.5]{"img/PB extraction 3"}
	}
	\only<7>{
		\includegraphics[scale=0.5]{"img/PB extraction 4"}
	}
	\only<8>{
		\includegraphics[scale=0.5]{"img/PB extraction 5"}
	}
	\only<9>{
		\includegraphics[scale=0.5]{"img/PB extraction all"}
	}
	
	
	\begin{itemize}
		\item<2-> multiple derivations can explain an ``observed'' phrase pair \\
		\item<9> we extract all of them once, irrespective of derivation
	\end{itemize}

}

\frame{
    \frametitle{Translation estimates}

	Number of times a (consistent) phrase pair is ``observed''
	$$c(\bar{f}, \bar{e})$$
	
	Relative frequency counting
	$$\phi(\bar{f}|\bar{e}) = \frac{c(\bar{f}, \bar{e})}{\sum_{\bar{f}'} c(\bar{f}', \bar{e})}$$
}

\frame{
    \frametitle{Features}

	\begin{itemize}
		\item language model
		\item forward translation probability $P(F|E)$
		\item backward translation probability $P(E|F)$
		\item forward and backward lexical smoothing
		\item word penalty 
		\item phrase penalty
		\item distance-based reordering model
		\item lexical reordering model
	\end{itemize}
	
}


\frame{
	\frametitle{Distance-based reordering}
	\begin{itemize}	
	\item exponential $\delta(d_k) = \alpha^{d_k}, \alpha < 1$
	\item $d_k=|\textrm{start}_k - \textrm{end}_{k-1} -1|$ 
	\end{itemize}
	~
	\pause
	
	\begin{center}
	\begin{tabular}{| l | l | p{1cm} | p{1cm} | p{1cm} | p{1cm} |}
	\hline
	& & \textcolor{red}{I} & \textcolor{red}{have} & \textcolor{red}{black} & \textcolor{red}{eyes} \\ \hline
	\textcolor{gray}{1}& \textcolor{blue}{J'} & \multicolumn{2}{c|}{\multirow{2}{*}{1}}  & & \\ \cline{1-2}\cline{5-6}
	\textcolor{gray}{2}& \textcolor{blue}{ai} & \multicolumn{2}{c|}{} & & \\ \hline
	\textcolor{gray}{3}& \textcolor{blue}{les} & & & & \multicolumn{1}{c|}{\multirow{2}{*}{3}} \\ \cline{1-5}
	\textcolor{gray}{4}& \textcolor{blue}{yeux} & & & &  \\ \hline
	\textcolor{gray}{5}& \textcolor{blue}{noirs} & & & \multicolumn{1}{c|}{2} & \\ \hline
	\end{tabular}
	\end{center}
		
	\begin{columns}
	
	\begin{column}{0.3\textwidth}
	\begin{itemize}
		\item $\bar{f}_1 = \textcolor{blue}{\text{J' ai}}$
		\item $\bar{e}_1 = \textcolor{red}{\text{I have}}$
		\item $\text{start}_1 = 1$
		\item $\text{end}_1 = 2$
	\end{itemize}
	\end{column}
	\begin{column}{0.3\textwidth}
	\begin{itemize}
		\item $\bar{f}_2 = \textcolor{blue}{\text{noirs}}$
		\item $\bar{e}_2 = \textcolor{red}{\text{black}}$
		\item $\text{start}_2 = 5$
		\item $\text{end}_2 = 5$
	\end{itemize}
	\end{column}
	\begin{column}{0.3\textwidth}
	\begin{itemize}
		\item $\bar{f}_3 = \textcolor{blue}{\text{les yeux}}$
		\item $\bar{e}_3 = \textcolor{red}{\text{eyes}}$
		\item $\text{start}_3 = 3$
		\item $\text{end}_3 = 4$
	\end{itemize}
	\end{column}
	
	\end{columns}
}

\frame{
	\frametitle{Conclusion}

	\begin{itemize}
	\item generative modelling requires approximations
	\item overfitting in fragment models (DOP)
	\item \citep{Koehn+2003:pbsmt} ignore segmentation:\\
		good feature choice in discriminative model
	\item reordering remains an issue 
	\end{itemize}
}

	\section{Decoding}
\subsection{Complexity}
\frame{
        \frametitle{Decoding}

        Disambiguation problem
        \begin{align*}
                \hat{E} 
                &= \argmax_E P(E)P(F|E) \\
                &= \argmax_E P(E) \sum_A P(F,A|E)
        \end{align*}
        {\small \hfill NP-complete \citep{Simaan:2002:complexity}}

        \pause

        ~

        Viterbi approximation
        \begin{align*}
                \hat{E} 
                &\approx \argmax_{E, A} P(E) P(F,A|E)\\
        \end{align*}

}

\frame{
	\frametitle{Viterbi decoding}
	The alignment space (or space of \emph{derivations})
	\begin{itemize}
		\item $O(2^n)$ segmentations\\
		\item $O(n!)$ permutations\\
		\item $O(t^n)$ substitutions\\
	\end{itemize}
	\pause
	~
	
	Packed representation using finite-state transducers
	$$O(n^2 \times \alert{2^n} \times t)$$
	\hfill NP-complete (TSP) \citep{Knight:1999:tsp,Zaslavskiy+2009:tsp} 
	
	
}

\frame{
	\frametitle{Complete model}
	
	\begin{align*}
		\alert{P(E)}P(F,S|E) 
		&= \alert{\prod_{j=1}^{|E|} \psi(e_j|e_{j - n + 1}^{j - 1})} \prod_{i=1}^{|S|} \textcolor{blue}{\phi(\bar{f}_i|\bar{e}_i)} \textcolor{Green}{\delta(\text{start}_i - \text{end}_{i-1} - 1)}
	\end{align*}

	Approximations:
	\begin{itemize}
		\item distortion limit $d$: $2^n \to 2^d$
		\item maximum phrase length $m$: $n^2 \to n \times m$
	\end{itemize}
	
	~

	\begin{itemize}

		\item alignment space $O(\textcolor{Green}{2^d} \times \textcolor{blue}{n \times m}\times t )$
		\item weighted derivations $O(\textcolor{Green}{2^d} \times \textcolor{blue}{n \times m} \times t \times \alert{|\Delta|^{k-1}})$ \\
		where $P(E)$ is a $k$-gram LM components over $\Delta^*$\\
		and $|\Delta| \propto t \times n$
	\end{itemize}

	\only<2->{
	\textbf<2->{This space is too large for exact inference}
	\begin{itemize}
		\item<3> pruning: beam search
	\end{itemize}
	}
	
}

\frame{
    \frametitle{Complexity}
	 \citep{Knight:1999:tsp}
	\begin{center}
       \only<1>{
                \includegraphics[width=0.7\textwidth]{"img/knight-tsp1"}
        }
        \only<2>{
                \includegraphics[width=0.5\textwidth]{"img/knight-tsp2"}
        }
	\end{center}
}

	
	\newcounter{finalframe}
	\setcounter{finalframe}{\value{framenumber}}
	
	
	{\setbeamertemplate{footline}{}
    \begin{frame}[plain]{Questions?}
    \end{frame}
  	}
	
	\section*{References}
	
	%
\frame[plain]{
	\frametitle{Earley intersection}
	
	\begin{footnotesize}
	\begin{align*}
	\textsc{Axioms} & \\
	& \drule{}{\itembrack{S' \ra \bullet S, q, q}}{q \in I} \\
	\textsc{Goal} & \\
	& \itembrack{S' \ra S \bullet, q, r} ~ q \in I \wedge r \in F\\
	\textsc{Scan} & \\
	& \drule{\itembrack{X \ra \alpha \bullet x \beta, q, s}}{\itembrack{X \ra \alpha x \bullet \beta}}{\angbrack{s, x, r} \in E}\\
	\textsc{Predict} & \\
	& \drule{\itembrack{X \ra \alpha \bullet Y \beta, q, r}}{\itembrack{Y \ra \bullet \gamma, r, r}}{Y \ra \gamma \in R} \\
	\textsc{Complete} & \\
	& \drule{\itembrack{X \ra \alpha \bullet Y \beta, q, s}\itembrack{Y \ra \gamma \bullet, s, r}}{\itembrack{X \ra \alpha Y_{s,r} \bullet \beta, q, r}}{X \neq S'} \\
	\textsc{Accept} & \\
	& \drule{\itembrack{S' \ra \bullet S, q, q}\itembrack{S \ra \gamma \bullet, q, r}}{\itembrack{S' \ra  S_{q,r} \bullet, q, r}}{r \in F} 
	\end{align*}
	\end{footnotesize}

	
}




	\frame[allowframebreaks]{ \frametitle{References}
        \bibliographystyle{plainnat}
        \bibliography{../../bib}
	}

	

	% Trick to discount regular frames from the total of backup frames
	\setcounter{framenumber}{\value{finalframe}}
	
	
\end{document}
